% Default course lecture note template by asp 
\documentclass[letterpaper]{article}
\usepackage[utf8]{inputenc}
\usepackage[T1]{fontenc}
\usepackage[english]{babel}
\usepackage[top=3cm, bottom=3cm, left=3.85cm, right=3.85cm]{geometry}
\usepackage[onehalfspacing]{setspace}
\usepackage{amsmath,amssymb,wasysym,amsthm}
\usepackage{graphicx}
\usepackage{mathptmx}
\usepackage{hyperref}
\usepackage{gitinfo}
\usepackage[usenames,dvipsnames]{color}

\title{Bitcoin Incentives}
\author{Andrew Poelstra}
\author{Bryan Bishop}
\date{\gitAuthorDate{} (commit \texttt{\gitAbbrevHash)}}

\begin{document}

\maketitle

\paragraph{Status.} This document is a work in progress and is missing large
sections of text. Bug reports should be communicated at
\url{https://github.com/kanzure/bitcoin-incentives}.

\section{Introduction}

"[It is hard to] figure out how to bootstrap an incentives model, because
there's no incentive except for the virtual currency payments, and yet the
value of the virtual currency payments only matters if it's secure, and it's
only secure if the incentive mechanism works....." - Andrew Miller

Here is an enumeration of various incentives that exist in the Bitcoin software
and p2p network. Some of these incentives are much stronger than others, but a
rating scheme or ordering scheme has yet to be devised. Additionally, some have
been observed in action in the wild, while others are more obscure and only
notable in very specific circumstances.

\section{Incentive}

An incentive is what happens when a giant entropy monster reaches out and grabs
you.

\section{Incentives}

\begin{enumerate}
\item Block subsidy incentive to miners for every block they mine.

\item Block transaction fees provide an incentive for miners to mine blocks.

\item Miner transaction fee incentive is somewhat limited by the byte size of
each transaction.

\item Miners have an incentive to publish blocks as quickly as possible because
of the lottery race they participate in against other miners who may also have
a valid block, or who may soon find a block.

\item Miners have an incentive to limit the size of their blocks so as to
minimize the time it takes for their block to saturate the network or reach
every node in the p2p network.

\item Various vague incentives about miners including any transaction, because
of the benefit of providing a working system, compared to rejecting the
majority of transactions and losing users that might pay transaction fees or
pay to other users that might pay transaction fees.

\item Miners have an incentive to include transactions with high fees not only
because "more money is good" but also because it removes the pool of available
transaction fees that other miners could benefit from.

\item Users have an incentive to include transaction fees to get transactions
processed faster than "normal". (?)

\item Incentives to provide accurate timestamps.

\item Incentives to relay transactions.

\item Incentives to relay blocks.

\item Incentives to verify transactions.

\item Incentives to maintain a mempool.

\item Incentives to not relay certain transactions.

\item Incentives to switch between forks.

\item Incentives to not switch between forks.

\item Incentives to broadcast blocks once found.

\item Incentives to provide initial block downloads.

\end{enumerate}

\section{Conclusion}

\begin{thebibliography}
\bibitem{satoshi} \url{https://www.bitcoin.org/bitcoin.pdf}
\end{thebibliography}

\end{document}
